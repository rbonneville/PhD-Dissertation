%  The dissertation abstract can only be 500 words.

Cancers begin through acquisition of genomic alterations in normal cells, leading to uncontrolled cell division. However, cancer cells continue to accumulate alterations throughout the disease course. Therefore cancer in a patient is a continually changing entity subject to microevolution. This leads to tumor heterogeneity, or genetic diversity within a cancer case. Heterogeneity significantly complicates cancer diagnosis and treatment, as different cancer cells in the same patient may behave and respond differently. Current research and clinical paradigms of cancer inspect the disease at one or a few fixed time points, for instance at a patient's oncologist visits. Furthermore, most clinical and research assays of cancer in patients analyze only the small subsets of cancer cells obtained through singular biopsies. Through studying mutation and heterogeneity, we can formulate models of cancer as the dynamic process that it is, and explore the temporal and spatial gaps in between the currently studied snapshots of cancer.

In this work, we first investigate microsatellite instability (MSI), a prolific mutational pattern present in subsets of human cancers. We introduce a new software algorithm, MANTIS, which uses next-generation sequencing data to quantify the accumulation of MSI rather than its mere presence or absence. Applying this to a large cohort of 11,139 cancers from 39 types, we identify MSI in 3.8\% of cases, including adrenocortical carcinoma where MSI had not previously been characterized. These findings have direct clinical applicability, as tumors with MSI are frequently sensitive to checkpoint inhibitor immunotherapy. Our analysis of such a diverse cohort highlights the potential of large-scale sequencing to identify and characterize rare phenotypes which may benefit subsets of patients.

Another means to assess changes in tumor heterogeneity over time is through subclonal modeling. Subclones provide a theoretical framework to approximate tumor heterogeneity and cancer microevolution through gradual accumulation of mutations. Here we infer tumor subclones and estimate their phylogeny in patients with cholangiocarcinoma, small-cell lung cancer, interdigitating dendritic cell sarcoma, and cancers with MSI\@. We leverage rapid research autopsy to obtain multiple high-quality tumor samples, and formulate subclonal models which identify shifts in cancer cell populations in response to treatment and at various metastatic sites. Furthermore, we provide extensions of the subclonal inference tool Canopy which provide ordering of mutations within phylogenetic branches, as well as quantification of MSI accumulation and microsatellite length distributions within tumor subclones. These findings demonstrate the power of subclonal modeling to provide insights into the timing and distribution of mutational events leading to tumor heterogeneity, and their contributions to treatment resistance and metastasis.

Taken together, these studies suggest large-cohort next-generation sequencing and refining subclonal analysis algorithms as future pathways for expanding the ``predictability horizon'' of cancer. Further understanding of the dynamics of cancer mutation, especially intratumor and interpatient heterogeneity, may permit for instance elucidation of the effects of treatments on tumor genetic composition, and prediction of genetic distinctions between primary and metastatic sites of disease. Such a view of cancer as a process hence has the potential to improve precision medicine care of this disease.
